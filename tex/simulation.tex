\chapter{モンテカルロシミュレーションにおける評価}\label{simulation}

\section{シミュレーション}

\subsection{シミュレーションの意義}

我々の実験では,複雑に物体が組み合わされており,我々が求めるイベントがどれくらいの頻度で発生しうるのかを計算するのは立体角や断面積の計算が入り組みかなり難しい.
そこでMonte Carloシミュレーションを用いて,実験レートを評価する.Geant4を用いてプラスチックシンチレータの中で停止せずに陽電子がターゲットに到達するか,そしてターゲットから発生したガンマ線がNaIシンチレータを光らせるかを確かめることができる.

\subsection{プラスチックシンチレータの厚み評価}
今回の実験で用いられたプラスチックシンチレータは0.15 mmの厚みのものである.この厚みが線源である

\subsection{実験レート評価}
