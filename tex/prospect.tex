\chapter{今後の課題}\label{prospect}

この章では,本実験での未達事項と新実験に向けてのアップデートについて簡単に述べる.

\section{$3\gamma$事象の積極的選別}
NaIシンチレータで取得される$\gamma$線のエネルギーを用いて$2\gamma$事象と$3\gamma$事象のさらなる選別が可能である.
511 keVの$\gamma$線を除くことで,効率よく$3\gamma$事象を選ぶことができる.
一方で,$2\gamma$事象であっても,NaIシンチレータでコンプトン散乱したイベントはエネルギーを用いて落とすことができない.
パラポジトロニウムが生成されたイベントは崩壊の時定数で区別出来るので,実質的なバックグラウンドとなり得るのはオルソポジトロニウムが生成され,スピン交換反応を起こしパラポジトロニウムの寿命より長い寿命で崩壊し,かつシンチレータでコンプトン散乱したイベントである.


\section{ランタイム}
卒業研究の限られた時間の中で,十分な統計量を得るためのランタイムを確保する事ができなかった.
1ヶ月程度の長期の測定を行うことで,より統計誤差を減らすことが期待される.

\section{Monte Carloシミュレーションと実験データの不一致}
実験データがでたらかく.
もう不一致って書いちゃってるけど.

\section{系統誤差の削減方針}
本実験での系統誤差の主たる原因について議論する.

\subsection{磁場強度}
本実験に用いた磁石については,第\ref{pmt}章で議論した.
有限要素法のシミュレーションによって,磁場強度についてはかなり詳細な理解がなされた.
一方で,内部に治具を全て導入した状態でのシミュレーションは行われていない.
支柱の大部分を占めるアルミニウムについては大きな影響がないと予想されるが,PMTを磁場から遮蔽するための鉄管や,治具固定の鉄ネジ等の影響は明らかではない.
これらについてもシミュレーションを行えば,より良い精度で磁場強度を求められる

\subsection{$2\gamma$,$3\gamma$比の推定}
まだしてない.
