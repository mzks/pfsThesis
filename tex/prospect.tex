\chapter{今後の課題(担当:水越)}\label{prospect}

本章では,今後の課題として,本研究の未達事項と実験を行った際に起こりうる問題の解決方法を議論する.

\section{磁場強度計算}
本実験に用いた磁石については,第\ref{PMT}章で議論した.
有限要素法のシミュレーションによって,磁場強度についてはかなり詳細な理解がなされた.
一方で,内部に治具を全て導入した状態でのシミュレーションは行われていない.
支柱の大部分を占めるアルミニウムについては大きな影響がないと予想されるが,PMTを磁場から遮蔽するための鉄管や,治具固定の鉄ネジ等の影響は明らかではない.
これらについてもシミュレーションを行えば,より良い精度で磁場強度を求められる.

\section{$3\gamma$事象の積極的選別}
NaIシンチレータで取得される$\gamma$線のエネルギーを用いて$2\gamma$事象と$3\gamma$事象のさらなる選別が可能である.
511 keVの$\gamma$線を除くことで,効率よく$3\gamma$事象を選ぶことができる.
一方で,$2\gamma$事象であっても,NaI(Tl)シンチレータでコンプトン散乱したイベントはエネルギーを用いて落とすことができない.
パラポジトロニウムが生成されたイベントは崩壊の時定数で区別出来るので,実質的なバックグラウンドとなり得るのはオルソポジトロニウムが生成され,スピン交換反応を起こしパラポジトロニウムの寿命より長い寿命で崩壊し,かつシンチレータでコンプトン散乱したイベントである.
このイベントはシリカゲルを焼いて水分を飛ばしたり,真空度を高めたりすることでより減らすことができると期待される.

また,本研究より2$\gamma$崩壊と3$\gamma$崩壊のスペクトルが得られている.
PMTの分解能を実測して導入すると,実験スペクトルとの比較が可能であり,2$\gamma$事象と3$\gamma$事象の比を求めることができる.

\section{まとめ}
本研究は時間的な制約により物理測定まで到達することができなかった.
しかしながら,実験装置はほぼ完成しており,数点既製の部品を購入することですぐに物理結果を得ることができるだろう.
本研究を礎として来年度以降の測定と物理結果を期待する.
