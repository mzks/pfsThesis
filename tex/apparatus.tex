\chapter{実験装置(担当:宮辺)}
\label{apparatus}

%\section{概要}

この章では本研究で用いた実験装置,
及び新たに導入したトリガー用のプラスチックシンチレータ(以下SCtrig)の評価について述べる.
%この章では本研究で用いた実験装置,
%及び今後の研究で導入予定のトリガー用のプラスチックシンチレータ(以下SCtrig)の評価と,
%実験装置の改良について述べる.


\section{寿命測定}

ポジトロニウムを形成するために使用する線源は$\mathrm{^{22}Na}$である.
$\mathrm{^{22}Na}$は,
$\beta^{+}$崩壊(式\ref{eq:beta})をする.
また$\beta^{+}$崩壊をしてできた励起状態の$\mathrm{^{22}Ne}$は,
1275 keVの$\gamma$線を放出し,基底状態の$\mathrm{^{22}Ne}$となる.(図\ref{fig:na})

\begin{equation}
\mathrm{^{22}Na} \to \mathrm{^{22}Ne} + \mathrm{e^{+}} + \nu_{\mathrm{e}}
\label{eq:beta}
\end{equation}

\begin{figure}[H]
\centering
\includegraphics[keepaspectratio,scale=0.4]{fig/ybm/na.pdf}
\caption{$\mathrm{^{22}Na}$の崩壊}
\label{fig:na}
\end{figure}

本研究では,トリガーとして陽電子がSCtrigを通過した信号を用い,
ポジトロニウムが崩壊し放出される$\gamma$線が検出された時間との差を測定することで,
寿命を計算する.
%本研究では,先発信号の1275 keVの$\gamma$線と,
%ポジトロニウムが崩壊し放出される$0\sim511 keV$の$\gamma$線が検出された時間差を測定することで,
%寿命を計算する.
ポジトロニウムの寿命$\tau$は
\begin{equation}
N(t) = N_{0} \exp( - \frac{t}{\tau})
\label{eq:lifefit}
\end{equation}
で定義される.
実験で得られた崩壊時間分布のヒストグラムを式(\ref{eq:lifefit})でフィッティングし,
寿命を得る.


\section{実験装置}

本研究ではトリガーのSCtrigから出た光を検出するために,
浜松ホトニクス製の光電子増倍管(PMT)アッセンブリR2248(図\ref{fig:pmtmini})を,
ポジトロニウムの崩壊による$\gamma$線を検出するために,
浜松ホトニクス製の光電子増倍管アッセンブリH6410(図\ref{fig:pmtbig})と
SCIONIX製のNaI(Tl)シンチレータ(図\ref{fig:naitl})を使用する.
NaI(Tl)シンチレータは,直径57 mm,長さ58 mmである.
PMT R2248の管径は9.8mm $\times$ 9.8mmの四角で,ソケットを含めた長さは100 mm,
ダイノード構造はラインフォーカス型である.
PMT H6410の管径は直径60 mm,長さが200 mm,
ダイノード構造はラインフォーカス型である.
%本研究ではトリガーの1.275 MeVと,
%ポジトロニウムの崩壊による$\gamma$線を検出するために,
%NaI(Tl)シンチレータ(図\ref{fig:naitl})と光電子増倍管(PMT)(図\ref{fig:pmt})を使用する.
%PMTは浜松ホトニクス製の光電子増倍管アッセンブリH6410で,
%ダイノード構造はラインフォーカス型である.

\begin{figure}[htbp]
\begin{minipage}{0.5\hsize}
\centering
\includegraphics[keepaspectratio,angle=90,scale=0.4]{fig/ybm/pmtmini.pdf}
\caption{PMT R2248の外形寸法図\newline[ref]https://www.hamamatsu.com/jp/ja/index.html}
\label{fig:pmtmini}
\end{minipage}
\begin{minipage}{0.5\hsize}
\centering
\includegraphics[keepaspectratio,angle=90,scale=0.4]{fig/ybm/pmtbig.pdf}
\caption{PMT H6410の外形寸法図\newline[ref]https://www.hamamatsu.com/jp/ja/index.html}
\label{fig:pmtbig}
\end{minipage}
\end{figure}

\begin{figure}[htbp]
\begin{minipage}{0.5\hsize}
\centering
\includegraphics[keepaspectratio,scale=0.35]{fig/ybm/naitl.pdf}
\caption{NaI(Tl)シンチレータ}
\label{fig:naitl}
\end{minipage}
\begin{minipage}{0.5\hsize}
\centering
\includegraphics[keepaspectratio,scale=0.35]{fig/ybm/naitl0.pdf}
\caption{NaI(Tl)シンチレータの概要図\newline[ref]http://www.crystals.saint-gobain.com/}
\label{fig:naitl0}
\end{minipage}
\end{figure}

%\begin{figure}[H]
%\centering
%\includegraphics[keepaspectratio,scale=0.4]{device1.pdf}
%\caption{実験装置の概要図}
%\label{fig:device1}
%\end{figure}

またポジトロニウムに磁場をかけるために図\ref{fig:mag}のような電磁石を使用する.
コイルの直径は370 mm,コイル間の距離は185 mm,
磁極の直径は100 mm,磁極間の距離は150 mmである.
定格6.0 Aの電流で,1.0 Tの磁場がかかる.

\begin{figure}[H]
\centering
\includegraphics[keepaspectratio,angle=90,scale=0.4]{fig/ybm/mag.pdf}
\caption{電磁石}
\label{fig:mag}
\end{figure}

実験装置は図ref{fig:device2},トリガー部分は図ref{fig:device3}のようになる.
$\mathrm{^{22}Na}$線源からの陽電子がコリメータを通り,SCtrigを通過し,
真空容器中のシリカエアロゲルでポジトロニウムを形成する.
SCtrigから出る光子はアクリルライトガイドを通り,
アクリルライトガイドに取り付けた2本のPMT R2248で検出される.
ポジトロニウムが崩壊し放出された$\gamma$線は,
周囲に4つ置かれたNaI(Tl)シンチレータとPMT H6410で検出される.
%第\ref{sec:PMT}
第-章で詳しく述べるが,すべてのPMTは磁場を遮蔽するための鉄管の中に入っている.

\begin{figure}[H]
\centering
\includegraphics[keepaspectratio,scale=0.4]{fig/ybm/device2.pdf}
\caption{実験装置の概要図}
\label{fig:device2}
\end{figure}

\begin{figure}[H]
\centering
\includegraphics[keepaspectratio,scale=0.4]{fig/ybm/device3.pdf}
\caption{トリガー部分の概要図}
\label{fig:device3}
\end{figure}

%実験装置の写真


\section{SCtrigの評価}

昨年度はトリガーに1.275 MeVの$\gamma$線を使用したが,
本研究では測定精度の向上のために,
トリガーに陽電子を検出するSCtrigを使用する.
%本研究ではトリガーに1.275 MeVの$\gamma$線を使用したが,
%今後測定精度の向上のために,
%トリガーに陽電子を検出するSCtrigを使用することを考えている.
SCtrigはサンゴバン社製のBC490,
ポリニトロトルエンでできており,厚さは150 $\si{\mu m}$,
発光量は9000 pkotons/MeVである.

\begin{figure}[H]
\centering
\includegraphics[keepaspectratio,scale=0.4]{fig/ybm/.pdf}
\caption{トリガーテスト装置(横から見た図)}
\label{fig:}
\end{figure}

\begin{figure}[H]
\centering
\includegraphics[keepaspectratio,scale=0.4]{fig/ybm/.pdf}
\caption{トリガーテスト装置(上から見た図)}
\label{fig:}
\end{figure}

\begin{figure}[H]
\centering
\includegraphics[keepaspectratio,scale=0.4]{fig/ybm/.pdf}
\caption{オシロスコープ}
\label{fig:}
\end{figure}

\begin{figure}[H]
\centering
\includegraphics[keepaspectratio,scale=0.4]{fig/ybm/.pdf}
\caption{ヒストグラム}
\label{fig:}
\end{figure}




