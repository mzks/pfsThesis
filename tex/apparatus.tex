\chapter{実験装置(担当:宮辺)}
\label{apparatus}

%\section{概要}

この章では本研究で用いた実験装置,
及び新たに導入したトリガー用のプラスチックシンチレータ(以下SCtrig)の評価について述べる.
%この章では本研究で用いた実験装置,
%及び今後の研究で導入予定のトリガー用のプラスチックシンチレータ(以下SCtrig)の評価と,
%実験装置の改良について述べる.


\section{寿命測定}

ポジトロニウムを形成するために使用する線源は$\mathrm{^{22}Na}$である.
$\mathrm{^{22}Na}$は,半減期が2.602 年であり,
$\beta^{+}$崩壊(式\ref{eq:beta}),電子捕獲により,$\mathrm{^{22}Ne}$になる.
この$Q$値は2842 keVである.\cite{ToRI}
また$\beta^{+}$崩壊をしてできた励起状態の$\mathrm{^{22}Ne}$は,
1275 keVの$\gamma$線を放出し,基底状態の$\mathrm{^{22}Ne}$となる.(図\ref{fig:na})

\begin{equation}
\mathrm{^{22}Na} \to \mathrm{^{22}Ne} + \mathrm{e^{+}} + \nu_{\mathrm{e}}
\label{eq:beta}
\end{equation}

\begin{figure}[H]
\centering
\includegraphics[keepaspectratio,scale=0.4]{fig/ybm/na.pdf}
\caption{$\mathrm{^{22}Na}$の崩壊\cite{ToI}}
\label{fig:na}
\end{figure}

昨年度までの研究では,トリガーとして1275 keVの$\gamma$線を使用していたが,
本研究では,トリガーとして陽電子がSCtrigを通過した信号を用い,
ポジトロニウムが崩壊し放出される$\gamma$線が検出された時間との差を測定することで,
寿命を計算する.
ポジトロニウムの寿命$\tau$は
\begin{equation}
N(t) = N_{0} \exp( - \frac{t}{\tau})
\label{eq:lifefit}
\end{equation}
で定義される.
実験で得られた崩壊時間分布のヒストグラムを式(\ref{eq:lifefit})でフィッティングし,
寿命を得る.


\section{実験装置}

本研究ではトリガーのSCtrigから出た光を検出するために,
浜松ホトニクス製の光電子増倍管(PMT)アッセンブリR2248(図\ref{fig:pmtmini})を,
ポジトロニウムの崩壊による$\gamma$線を検出するために,
SCIONIX製のNaI(Tl)シンチレータ(図\ref{fig:naitl})と,
浜松ホトニクス製の光電子増倍管アッセンブリH6410(図\ref{fig:pmtbig})を使用する.
NaI(Tl)シンチレータは,直径57 mm,長さ58 mmの円筒形である.
PMT R2248の管径は9.8mm $\times$ 9.8mmの四角で,ソケットを含めた長さは100 mm,
ダイノード構造はラインフォーカス型である.
PMT H6410の管径は直径60 mm,長さが200 mm,
ダイノード構造はラインフォーカス型である.
PMTは完全な暗中にあるときでも微小な電流を出力している.
これを暗電流といい,測定する前にPMTを数十分程度暗中に放置することで減少させることができる.
%本研究ではトリガーの1.275 MeVと,
%ポジトロニウムの崩壊による$\gamma$線を検出するために,
%NaI(Tl)シンチレータ(図\ref{fig:naitl})と光電子増倍管(PMT)(図\ref{fig:pmt})を使用する.
%PMTは浜松ホトニクス製の光電子増倍管アッセンブリH6410で,
%ダイノード構造はラインフォーカス型である.

\begin{figure}[htbp]
\begin{minipage}{0.5\hsize}
\centering
\includegraphics[keepaspectratio,scale=0.35]{fig/ybm/naitl.pdf}
\caption{NaI(Tl)シンチレータ}
\label{fig:naitl}
\end{minipage}
\begin{minipage}{0.5\hsize}
\centering
\includegraphics[keepaspectratio,scale=0.35]{fig/ybm/naitl0.pdf}
	\caption{NaI(Tl)シンチレータの概要図\cite{nai}}
\label{fig:naitl0}
\end{minipage}
\end{figure}

\begin{figure}[htbp]
\begin{minipage}{0.5\hsize}
\centering
\includegraphics[keepaspectratio,angle=90,scale=0.4]{fig/ybm/pmtmini.pdf}
	\caption{PMT R2248の外形寸法図\cite{pmtshape}}
\label{fig:pmtmini}
\end{minipage}
\begin{minipage}{0.5\hsize}
\centering
\includegraphics[keepaspectratio,angle=90,scale=0.4]{fig/ybm/pmtbig.pdf}
	\caption{PMT H6410の外形寸法図\cite{pmtshape}}
\label{fig:pmtbig}
\end{minipage}
\end{figure}


%\begin{figure}[H]
%\centering
%\includegraphics[keepaspectratio,scale=0.4]{device1.pdf}
%\caption{実験装置の概要図}
%\label{fig:device1}
%\end{figure}

またポジトロニウムに磁場をかけるために図\ref{fig:mag}のような電磁石を使用する.
コイルの直径は370 mm,コイル間の距離は185 mm,
磁極の直径は100 mm,磁極間の距離は150 mmである.
定格6.0 Aの電流で,0.1 Tの磁場がかかる.

\begin{figure}[H]
\centering
\includegraphics[keepaspectratio,angle=90,scale=0.4]{fig/ybm/mag.pdf}
\caption{電磁石}
\label{fig:mag}
\end{figure}

実験装置は図\ref{fig:device2},トリガー部分は図\ref{fig:device3}のようになる.
$\mathrm{^{22}Na}$線源からの陽電子がコリメータを通り,SCtrigを通過し,
真空容器中のシリカエアロゲルでポジトロニウムを形成する.
SCtrigから出る光子はアクリルライトガイドを通り,
アクリルライトガイドに取り付けた2本のPMT R2248で同時計測し,
ポジトロニウムが崩壊し放出された$\gamma$線は,
周囲に4つ置かれたNaI(Tl)シンチレータとPMT H6410で検出される.
トリガーに陽電子を用いることで,
陽電子がシリカエアロゲルに到達し,ポジトロニウムを形成していることを確かめ,
またSCtrigに取り付けた2本のPMTによる同時計測により,
暗電流のレートを減らし,バックグラウンドを低減させることが,SCtrigを導入する目的である.
第\ref{PMT}章で詳しく述べるが,すべてのPMTは磁場を遮蔽するための鉄管の中に入っている.

\begin{figure}[H]
\centering
\includegraphics[keepaspectratio,scale=0.4]{fig/ybm/device2.pdf}
\caption{実験装置の概要図}
\label{fig:device2}
\end{figure}

\begin{figure}[H]
\centering
\includegraphics[keepaspectratio,scale=0.4]{fig/ybm/device3.pdf}
\caption{トリガー部分の概要図}
\label{fig:device3}
\end{figure}

%実験装置の写真


\section{SCtrigの評価}

昨年度はトリガーに1.275 MeVの$\gamma$線を使用したが,
本研究では測定精度の向上のために,
トリガーに陽電子を検出するSCtrigを使用する.
SCtrigはサンゴバン社製のBC490,
ポリニトロトルエンでできており,厚さは150 $\si{\micro m}$,
発光量は9000 photons/MeVである.

図\ref{fig:coinci1},\ref{fig:coinci2}のように,
フィルム状のミラーでSCtrigを挟み,2本のPMTをオプティカルグリスで接着する.
これは2本のPMTで同時計測することにより,
暗電流(\ref{fig:oscillo2})のレートを低減することが目的である.
またSCtrigの後方にもうひとつプラスチックシンチレータが取り付けられたPMT H6410を置く.
これは陽電子がSCtrigを通過していることを確かめることが目的である.
この結果はオシロスコープで図\ref{fig:oscillo},\ref{fig:oscillo1}のようになった.

\begin{figure}[H]
\begin{minipage}{0.5\hsize}
\centering
\includegraphics[keepaspectratio,scale=0.4]{fig/ybm/coinci1.pdf}
\caption{トリガーテスト装置(横から見た図)}
\label{fig:coinci1}
\end{minipage}
\begin{minipage}{0.5\hsize}
\centering
\includegraphics[keepaspectratio,scale=0.4]{fig/ybm/coinci2.pdf}
\caption{トリガーテスト装置(上から見た図)}
\label{fig:coinci2}
\end{minipage}
\end{figure}

\begin{figure}[htbp]
\centering
\includegraphics[keepaspectratio,scale=0.3]{fig/ybm/oscillo2.pdf}
	\caption[暗電流]{暗電流\newline
信号はSCtrigに取り付けたPMTのもの.トリガーは下のチャンネル.}
\label{fig:oscillo2}
\end{figure}

\begin{figure}[htbp]
%\begin{minipage}{0.5\hsize}
\centering
\includegraphics[keepaspectratio,scale=0.3]{fig/ybm/oscillo.pdf}
	\caption[2本のPMTの同時計測]{2本のPMTの同時計測\newline 上2本はSCtrigに取り付けたPMTの信号,下は後方に置いたPMTの信号}
\label{fig:oscillo}
\end{figure}
	\begin{figure}[htbp]
%\end{minipage}
%\begin{minipage}{0.5\hsize}
\centering
\includegraphics[keepaspectratio,scale=0.3]{fig/ybm/oscillo1.pdf}
\caption{図\ref{fig:oscillo}のシングルショット}
\label{fig:oscillo1}
%\end{minipage}
\end{figure}

図\ref{fig:oscillo}より陽電子がSCtrigを突き抜けていることが確かめられた.
また図\ref{fig:oscillo1}の波形を積分し式(\ref{eq:charge})で電荷に直したヒストグラムは図\ref{fig:charge}のようになった.

\begin{equation}
Q = \frac{V}{R}
\label{eq:charge}
\end{equation}

ここで$Q$ [C]は電荷,$V$ [V]はオシロスコープの波形を積分した電圧,$R [\Omega]$は終端抵抗で今は50 $\Omega$である.

\begin{figure}[htbp]
\centering
\includegraphics[keepaspectratio,angle=270,scale=0.7]{fig/ybm/charge.pdf}
\caption{ヒストグラム}
\label{fig:charge}
\end{figure}

図\ref{fig:oscillo2}より暗電流の信号の波形の高さは数百 mVであり,
図\ref{fig:oscillo1}より陽電子による信号の波形の高さは数十 mV,幅は10 nsであることがわかる.
図\ref{fig:oscillo2}の典型的な信号の波形は20 mV程度であり,
これを1個の光電子による信号とみなす.
図\ref{fig:coinci1},\ref{fig:coinci2}の治具の効率は,
SCtrigの屈折率や立体角から3 \%と見積もった.
PMTの量子効率は25 \%である.
また$\mathrm{^{22}Na}$から出る陽電子が,SCtrigを通過するときのエネルギー損失は約40 keVであり,
そのエネルギーによって約360 個の光子が放出される.
\begin{equation}
360 \times 0.03 \times 0.25 \simeq 3
\label{eq:pe}
\end{equation}
より約3個の光子がPMTに入射していることがわかる.
以上より信号の電荷は
\begin{equation}
	Q = \frac{\frac{1}{2} \times 20 [\si{mV}] \times 10[\si{ns}]}{50 \Omega} \times 3 = 6 [\si{pC}]
\label{eq:charge1}
\end{equation}
と計算することができる.
この計算は図\ref{fig:charge}と矛盾していない.
図\ref{fig:charge}を見ると,信号の電荷は大きくても40 pC程度であり,
対応する電圧は400 mVであり,信号は暗電流と同程度かそれ以下の電圧であるので,
ディスクリミネータの閾値による事象選別は不可能である.
以上のことから,トリガーにはディスクリミネータによる閾値を設定せずに, 2本のPMTの同時計測信号を使用することとした.

