\chapter{磁場中でのPMTの動作(担当:井口)}\label{PMT}


\section{動機}
磁場中のポジトロニウムの寿命測定実験(本実験)では,2インチPMTとSCtrig用のPMT R2248(R2248)\cite{pmtR2248}を磁場中で使用する.
この章で述べる実験(本測定)では,PMTを磁場中で使用した場合のゲインの変化を,標準$\gamma$線源のスペクトルを測定することで評価した.また鉄管でPMTを覆い同様の測定をおこなうことで磁場の遮蔽を試みた場合のゲインの変化を評価した.


\section{電磁石}
図\ref{magfigure}のように,本実験に用いる電磁石の磁極の直径は100 mm,磁極間の距離は150 mmである.以降では図\ref{magphoto}で示すように,電磁石のN極からS極へ向かう方向を$z$方向とし磁極間の中点を$z=0$ cmと定義する.さらに,磁極の半径方向を$r$方向とし磁極の中心を$r=0$ cmと定義する.
\begin{figure}[tbp]
	\centering
	\includegraphics[width=13cm]{fig/iguchi/magnetfigure.pdf}
	\caption{電磁石図}
	\label{magfigure}
\end{figure}

\begin{figure}[tbp]
	\centering
	\includegraphics[width=8cm]{fig/iguchi/magnetphoto.pdf}
	\caption{電磁石の写真}
	\label{magphoto}
\end{figure}

\newpage
電磁石に定格6.0 Aの電流を流し,ガウスメータを用いて,$z=0$ cmでの$z$方向の磁場の強さと$r$方向の関係を測定した.図\ref{maggraph01}は測定結果をプロットしたものである.電磁石のコイルの半径は185 mmであるため,磁場の強い$r=20$ ㎝以下は2 cm毎,磁場の大きさの変化が少ない$r=20$ cm以上からは5 cm毎に測定した.
磁極の中心の磁場の大きさは約90 mTであった.
\begin{figure}[h!]
	\centering
	\includegraphics[width=15cm]{fig/iguchi/maggraph01.pdf}
	\caption{$z$方向磁場の$r$依存}
	\label{maggraph01}
\end{figure}


\newpage
\section{2インチPMT}
本実験では浜松ホトニクス社製の光電子増倍管アッセンブリH6410\cite{pmtH6410}を使用する(図\ref{PMTphoto}).

\begin{table}[tbp]
	\centering
	 \begin{tabular}{cccccccc}\hline
	型名& 管径 & 外径 & 長さ & 最大電圧 & 印加電圧 & ゲイン & ダイノード構造 \\ \hline \hline
	H6410 & 2 インチ & 60 mm & 200 mm & -2000 V & -1850 V & $3.0\times10{^{6}}$ &ラインフォーカス型/10段 \\ \hline
	\end{tabular}
	  \caption{2インチPMTデータ表}
\end{table}

\begin{figure}[tbp]
  \begin{center}
    \begin{tabular}{c}
    %1
      \begin{minipage}[t]{0.5\hsize}    
        \begin{center}
          \includegraphics[width=6.5cm]{fig/iguchi/PMTphoto.jpg}
	\caption{2インチPMT H6410}
	\label{PMTphoto}
     \end{center}
    \end{minipage}
    %2
      \begin{minipage}[t]{0.5\hsize}    
        \begin{center}
          \includegraphics[width=6.5cm]{fig/iguchi/PMTinner.pdf}
         \caption{ラインフォーカス型の内部構造}
         \label{PMTinner}
         \end{center}
        \end{minipage}
     
     \end{tabular}
    \end{center}
 \end{figure}

PMTを磁場中で使用すると,図\ref{PMTinner}のように,光電子がPMT内部を進む間にローレンツ力を受けて軌道が曲がり,ダイノードに届かなくなることでゲインが著しく低下する.


\subsection{2インチPMTの測定}
2インチPMTの光電面にNaI(Tl)シンチレータ(直径:57 mm,長さ:58 mm)図\ref{NaIscinti}を取り付け,$\ce{^{60}Co}$と光電面の距離を70 mmに固定し,磁場なしの場合と磁場ありの場合でゲインを測定した.また図\ref{soutizu1}のように磁極の中心からPMTの光電面までの距離を$r$方向の距離とする.
 
\begin{figure}[tbp]
  \begin{center}
    \begin{tabular}{c}
    %1
      \begin{minipage}[t]{0.5\hsize}    
        \begin{center}
          \includegraphics[width=6.5cm]{fig/iguchi/2inchsokutei.pdf}
	\caption{2インチPMTの測定装置}
	\label{2inchsokutei}
     \end{center}
    \end{minipage}
    %2
      \begin{minipage}[t]{0.5\hsize}    
        \begin{center}
          \includegraphics[width=6.5cm]{fig/iguchi/soutizu1.pdf}
         \caption{装置図}
	\label{soutizu1}
         \end{center}
        \end{minipage}
     
     \end{tabular}
    \end{center}
 \end{figure}

\begin{figure}[tbp]
	\centering
	\includegraphics[width=5cm]{fig/iguchi/NaIscinti.jpg}
	\caption{NaI(Tl)シンチレータ}
	\label{NaIscinti}
\end{figure}

 
\subsection{2インチPMTの測定結果}\label{noFe}

\begin{figure}[tbp]
	\centering
	\includegraphics[angle=-90,width=10cm]{fig/iguchi/121450of.pdf}
	\caption{磁場なしのヒストグラム}
	\label{histoff}
\end{figure}

\begin{figure}[tbp]
	\centering
	\includegraphics[angle=-90,width=10cm]{fig/iguchi/121550on.pdf}
	\caption{$r=50$ cmのヒストグラム}
	\label{hist50}
\end{figure}

\begin{figure}[tbp]
	\centering
	\includegraphics[angle=-90,width=10cm]{fig/iguchi/121635on.pdf}
	\caption{$r=35$ cmのヒストグラム}
	\label{hist35}
\end{figure}

以下の測定では豊伸電子社製ADC V005を用いた.図\ref{histoff}が磁場なしの場合で取得した$\ce{^{60}Co}$のスペクトルである.図\ref{hist50}は磁場あり,$r=50$ cmで取得したスペクトルである.磁場なしの場合と同じように$\ce{^{60}Co}$の1.17 MeV,1.33 MeVの2つのピークが見える.$r=35$ cm(1.22 mT)のときに図\ref{hist35}のようにゲインが大幅に低下し,1.17 MeVのピークが見えにくくなっている.これはPMTの暗電流のノイズを落とすためディスクリミネータのしきい値を高くした結果,1.17 MeVのピークがしきい値によって切られためである.$r=25$ cm(4.82 mT)では信号が完全に消失した.
図\ref{plot2inchoff}は縦軸に2インチPMTの磁場中のゲインを,横軸にr方向の距離をとってグラフにした.ただしゲインとは,スペクトルの2つのピークをそれぞれガウシアンでフィットしたものの平均値である.$\ce{^{60}Co}$の2つのピークの形は同じだと考えて,バックグラウンドは一次関数で表せると考えた.縦軸は磁場なしのゲインで規格化した.$r=35$ cm以下では,ピークがフィット出来なかったためプロットしていない.

\begin{figure}[tbp]
	\centering
		\includegraphics[angle=-90,width=15cm]{fig/iguchi/plot2inchPMT.pdf}
	\caption{2インチPMTの磁場中でのゲイン変化}
	\label{plot2inchoff}
\end{figure}
2インチPMTは$r=35$ cm,$z$方向磁場の強さが1.22 mTで使用出来なくなると結論付ける.

\section{2インチPMTを磁場遮蔽した測定}
2インチPMTを磁場中で使用した場合,$r=35$ cmでスペクトルを取得できなくなる.本実験では2インチPMTを$r=12$ cmに設置するので,磁場を遮蔽する必要がある.磁場を遮蔽することで2インチPMTを使用できるか評価するため,2インチPMTを鉄管\ref{ironpipe}に挿入して先述と同様の測定をおこなった.
\begin{figure}[tbp]
	\centering
		\includegraphics[width=10cm]{fig/iguchi/ironpipe.JPG}
	\caption{鉄管}
	\label{ironpipe}
\end{figure}

\begin{table}[tbp]
	\centering
	 \begin{tabular}{cccc} \hline
		素材 & 外径 & 内径 & 長さ \\ \hline \hline
		軟鉄 & 76.3 mm & 62.3 mm & 270 mm \\ \hline
	\end{tabular}
	  \caption{鉄管データ表}
	  \label{ironpipedata}
\end{table}

磁力線が鉄管端から内部に入り込む可能性を考慮して,鉄管内部図\ref{PMTinFe}のように鉄管端とPMTの光電面との距離が70 mmになるように取り付けた.
鉄は強磁性体であり,磁力線は強磁性体に吸収されるため,鉄管内部は磁力線が通らなくなり磁場が遮蔽される.

\begin{figure}[tbp]
  \begin{center}
    \begin{tabular}{c}
    %1
      \begin{minipage}[tbp]{0.6\hsize}    
        \begin{center}
          \includegraphics[width=6.5cm]{fig/iguchi/PMTinFe.pdf}
	\caption{鉄管内部図}
	\label{PMTinFe}
     \end{center}
    \end{minipage}
    %2
      \begin{minipage}[tbp]{0.4\hsize}    
        \begin{center}
          \includegraphics[width=6.5cm]{fig/iguchi/jibakyusyu.jpg}
         \caption{強磁性体によって吸収される磁場\cite{jibashield}}
	  \label{jibakyusyu}
         \end{center}
        \end{minipage}
     
     \end{tabular}
      \end{center}
   \end{figure}
   
\begin{figure}[tbp]
	\centering
		\includegraphics[width=7cm]{fig/iguchi/2inchinFe.jpg}
	\caption{2インチPMTを磁場遮蔽した測定装置}
	\label{2inchinFe}
\end{figure}


\subsection{2インチPMTを磁場遮蔽した測定結果}

\begin{figure}[tbp]
	\centering
		\includegraphics[clip,angle=-90,width=10cm]{fig/iguchi/122350fe.pdf}
	\caption{2インチPMTを磁場遮蔽した$r=50$ cmのヒストグラム}
	\label{hist50fe}
\end{figure}
\begin{figure}[tbp]
	\centering
		\includegraphics[clip,angle=-90,width=10cm]{fig/iguchi/12237fe.pdf}
	\caption{2インチPMTを磁場遮蔽した$r=7$ cmのヒストグラム}
	\label{hist7fe}
\end{figure}

\begin{figure}[tbp]
	\centering
		\includegraphics[angle=-90,width=15cm]{fig/iguchi/bigPMTfit.pdf}
	\caption{2インチPMTを磁場遮蔽したときのゲイン変化}
	\label{bigPMTfit}
\end{figure}

鉄管なしの場合と同様にADCを用いて$\ce{^{60}Co}$のスペクトルを取得した.
鉄管に入れた場合,$r=7$ cmまで図\ref{hist7fe}のように$\ce{^{60}Co}$の1.17 MeV,1.33 MeVの二つのピークが見える.
図\ref{bigPMTfit}は,第\ref{noFe}節と同様の手法でピークをフィットして得たゲインを規格化してプロットしたものである.$r=20$ cmまでゲインがほとんど低下せず,$r=20$ cm以下でもゲインの低下は一割程度である.
本実験において2インチPMTは$r=12$ cmに設置するため,鉄管に挿入し磁場を遮蔽することで,本実験でも使用可能であることが示された.


\newpage
\section{3/8インチPMT}
四角型のPMT\hspace{3pt}R2248\cite{pmtR2248}は,本実験でSCtrig用に使用するため,第\ref{apparatus}章の図\ref{fig:device2}のように2インチPMTより磁極に近い位置に,光電面が$r=1.5$ cmとなるように設置する.磁場の影響の大きさはダイノード構造に主に関係するため,以下の測定ではPMT R2248と同じダイノード構造を持った3/8インチPMT\cite{pmtH3164-10}を使用した.

\begin{figure}[tbp]
	\centering
		\includegraphics[width=10cm]{fig/iguchi/miniPMT.jpg}
	\caption{3/8インチPMT H3164-10}
	\label{3/8inch}
\end{figure}

\begin{table}[tbp]
	\centering
	
	  \begin{tabular}{cccccccc} \hline
		型名& 管径 & 外径 & 長さ & 最大電圧 & 印加電圧 & ゲイン & ダイノード構造 \\ \hline \hline
		H3164-10 & 3/8 インチ & 10.5 mm & 45 mm & -1500 V & -1400 V & $1.0\times10{^{6}}$ &ラインフォーカス型/8段 \\ \hline
	\end{tabular}
	  \caption{3/8インチPMTデータ表}
	  \label{3/8inchPMT}
\end{table}

\begin{table}[tbp]
	\centering
	  \begin{tabular}{cccccccc} \hline
		型名& 管径 & 外径 & 長さ & 最大電圧 & 印加電圧 & ゲイン & ダイノード構造 \\ \hline \hline
		R2248 & 8 mm(四角型) & 9.8 mm(四角型)& 45 mm & -1500 V & - & $1.1\times10{^{6}}$ &ラインフォーカス型/8段 \\ \hline
	\end{tabular}
	  \caption{四角型PMTデータ表}
\end{table}


\subsection{3/8インチPMTの測定}
2インチPMTの動作確認実験と同様の方法で,$\ce{^{22}Na}$と3/8インチPMTの光電面との距離を100 mmで固定し,鉄管に入れた場合と入れない場合でゲインを測定した.
3/8インチPMTは暗電流由来のノイズが多く線源のスペクトルが見えにくかったため,一つのNaI(Tl)シンチレータに二つの3/8インチPMTを取り付け,それらの信号を同時計測することでノイズとシンチレータの発光の信号を区別した.

\begin{figure}[tbp]
  \begin{center}
    \begin{tabular}{c}
    %1
      \begin{minipage}[tbp]{0.4\hsize}    
        \begin{center}
          \includegraphics[width=6.5cm]{fig/iguchi/PPMT.jpg}
	\caption{3/8インチPMTの同時計測用治具}
	\label{PPMT}
     \end{center}
    \end{minipage}
    %2
      \begin{minipage}[tbp]{0.6\hsize}    
        \begin{center}
          \includegraphics[width=10cm]{fig/iguchi/miniset.pdf}
         \caption{3/8インチPMTの装置図}
	\label{miniset}
         \end{center}
        \end{minipage}
     
     \end{tabular}
    \end{center}
   \end{figure}
   
また鉄管は2インチPMTの測定と同一のものを使用し,線源はスペクトルのピークの位置が区別しやすい標準線源として$\ce{^{22}Na}$を使用した.


\subsection{3/8インチPMTの測定結果}

\begin{figure}[tbp]
	\centering
		\includegraphics[angle=-90,width=10cm]{fig/iguchi/minicoincidence2.pdf}
	\caption{磁場なしの場合の3/8インチPMTのヒストグラム}
	\label{histminicoincidence2}
\end{figure}

\begin{figure}[tbp]
	\centering
		\includegraphics[angle=-90,width=10cm]{fig/iguchi/minicoout22.pdf}
	\caption{3/8インチPMTを磁場中の$r=22$ cmで使用したヒストグラム}
	\label{histminicoout22}
\end{figure}

\begin{figure}[tbp]
	\centering
		\includegraphics[angle=-90,width=10cm]{fig/iguchi/minicoin22.pdf}
	\caption{3/8インチPMTを磁場遮蔽した$r=50$ cmのヒストグラム}
	\label{histminicoin22}
\end{figure}

\begin{figure}[tbp]
	\centering
		\includegraphics[angle=-90,width=10cm]{fig/iguchi/minicoin21.pdf}
	\caption{3/8インチPMTを磁場遮蔽した$r=6$ cmのヒストグラム}
	\label{histminicoin21}
\end{figure}

\begin{figure}[tbp]
	\centering
		\includegraphics[angle=-90,width=10cm]{fig/iguchi/miniPMTgainG.pdf}
	\caption{3/8インチPMTの磁場中でのゲイン変化}
	\label{miniPMTgainG}
\end{figure}

2インチPMTの測定と同様に豊伸電子社製ADC V005を用いて$\ce{^{22}Na}$のスペクトルを取得した.
図\ref{histminicoincidence2}が磁場なしで取得した$\ce{^{22}Na}$のスペクトルである.
図\ref{miniPMTgainG}は第\ref{noFe}節と同様の手法でピークをフィットして得たゲインを規格化してプロットしたものである.
1.275MeVのピークは図\ref{histminicoout22}のようにガウシアンと二次関数でフィットした.磁場遮蔽しない場合,$r=30$ cm(2.36 mT)まではゲインが低下しないが,$r=25$ cm(4.82 mT)付近からゲインが低下し,図\ref{histminicoout22}のように$r=22$ cm(10 mT以下)でゲインが低下しピークが見えなくなった.
鉄管に入れた場合,$r=6$ cmまでゲインが低下せず,図\ref{histminicoin21}のヒストグラムのようにピークがはっきりと見える.
測定結果から3/8インチPMTは信号を同時計測し,鉄管に入れて磁場を遮蔽することで$r=6$ cmまで使用できることがわかった.

\section{磁場シミュレーション}
本実験では図\ref{honjikken}のように,PMT R2248を外径21.7 mm,内径16.1 mmの鉄管に挿入し,3/8インチPMTを$r=1.5$ cmに設置する.

\begin{figure}[tbp]
	\centering
		\includegraphics[width=10cm]{fig/iguchi/honjikken.pdf}
	\caption{本実験でのセットアップ}
	\label{honjikken}
\end{figure}

本実験をおこなう前に,上のセットアップでもPMT R2248が使用出来るかを確認する必要がある.そのため,磁場シミュレーションソフト(Femtet)を使用して電磁石の磁場を再現し,PMT R2248を挿入する鉄管も同様に再現して,鉄管内部の磁場をシミュレーションした.

\subsection{磁場シミュレーション結果}
\begin{figure}[tbp]
	\centering
		\includegraphics[width=16cm]{fig/iguchi/magnetgraph2.pdf}
	\caption{Femtetで再現した磁場と実測定した磁場}
	\label{magnetgraph2}
\end{figure}

図\ref{magnetgraph2}から確認できるように,Femtetで再現した$z$方向磁場と実際の磁場の測定結果を比較すると,ほぼ一致していることからFemtetの磁場シュミレーションは妥当であることがわかる.

\begin{figure}[tbp]
	\centering
		\includegraphics[width=15cm]{fig/iguchi/Femtetsaigengraph.pdf}
	\caption{Femtetで鉄管を含めて再現した磁場}
	\label{Femtetsaigengraph}
\end{figure}

図\ref{Femtetsaigengraph}の,鉄管内部の$z$方向磁場のシミュレーション結果から,本実験でPMT R2248の光電面が設置される$r=1.5$ cmでの磁場の強さは約1.5 mTである.3/8インチPMTが使用可能であった$r=30$ cmでの磁場の強さは2.36 mT,また3/8インチPMTが使用不可能になった磁場の強さは約10 mTである.これらの結果を比較すると,シミュレーションで再現した鉄管内部の磁場の方が弱く,鉄管で磁場を十分に遮蔽できると考えられるため,PMT R2248は管径21.7 mmの鉄管に入れることで磁場中でも使用可能であると示された.

\section{まとめ}
2インチPMTは鉄管に挿入し磁場を遮蔽することで$r=7$ cmまで使用可能であったため,本実験で設置する予定である$r=12$ cmでも鉄管で磁場遮蔽することで使用可能であると結論付ける.また3/8インチPMTは2つのPMTの信号を同時計測し鉄管に挿入することで$r=6$ cmまで使用可能であった.本実験でSCtrig用のPMT R2248を挿入する鉄管の内部磁場を磁場シュミレーションで再現したものと,実際の測定結果を比較することで,PMT R2248は本実験で設置する$r=1.5$ cmでも鉄管で磁場遮蔽することで使用可能であると結論付ける.



