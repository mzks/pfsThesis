\chapter{磁場中でのPMTの動作確認(担当:井口)}\label{PMT}


\section{動機}
磁場中のポジトロニウムの寿命測定実験(本実験)では,2インチPMTとSCtrig用の四角型PMT(R2248)\cite{pmtR2248}を磁場中で使用する.PMTは一般に磁場中ではゲインが低下してしまうため,あらかじめ磁場中でPMTの動作を確認する必要がある.この章で述べる実験では,PMTを磁場中で使用した場合のゲインの変化を,$\gamma$線源のスペクトルを測定することで確認した.また鉄管でPMTを覆い同様の測定をおこなうことで磁場の遮蔽を試みた場合のゲインの変化を確認した.


\section{電磁石}
この実験では,本実験と同じ電磁石を用いる.2章でも述べたように,磁極の直径は100 mm,磁極間の距離は150 mmである.またこの章では,電磁石のN極からS極へ向かう方向を$z$方向とし,磁極間の中点を$z=0$ cmとする.さらに磁極の半径方向を$r$方向とし,磁極の中心を$r=0$ cmとする.
\begin{figure}[h!]
	\centering
	\includegraphics[width=13cm]{fig/iguchi/magnetfigure.pdf}
	\caption{電磁石図}
	\label{magfigure}
\end{figure}

\begin{figure}[h!]
	\centering
	\includegraphics[width=8cm]{fig/iguchi/magnetphoto.pdf}
	\caption{電磁石の写真}
	\label{magphoto}
\end{figure}

\newpage
はじめに電磁石に定格6.0 Aの電流を流し,ガウスメータを用いて,$z=0$ cmでの$z$方向の磁場の強さと$r$方向の関係を測定した.電磁石のコイルの半径は185 mmであるため,磁場の強い$r=20$ ㎝以下は2 cm毎,磁場の大きさの変化が少ない$r=20$ cm以上からは5 cm毎に測定した.
磁極の中心の磁場の大きさは約90 mTだった.
\begin{figure}[h!]
	\centering
	\includegraphics[width=15cm]{fig/iguchi/maggraph01.pdf}
	\caption{$z$方向磁場の$r$依存}
	\label{maggraph01}
\end{figure}

%\begin{itemize}
 %      \item 電磁石のN極からS極の方向を$z$方向とし,磁極間の中点を$z=0$ cmとする.
  %     \item 磁極の半径方向を$r$方向とし,磁極の中心を$r=0$ cmとする.
%\end{itemize}


\newpage
\section{2インチPMT}
今回の実験で使用した2インチPMTは,本実験で使用する予定である浜松ホトニクス製の光電子増倍管アッセンブリH6410\cite{pmtH6410}である.

\begin{table}[htb]
	\centering
	 \begin{tabular}{cccccccc}\hline
	型名& 管径 & 外径 & 長さ & 最大電圧 & 印加電圧 & ゲイン & ダイノード構造 \\ \hline \hline
	H6410 & 2 インチ & 60 mm & 200 mm & -2000 V & -1850 V & $3.0\times10{^{6}}$ &ラインフォーカス型/10段 \\ \hline
	\end{tabular}
	  \caption{2インチPMTデータ表}
\end{table}

\begin{figure}[h]
  \begin{center}
    \begin{tabular}{c}
    %1
      \begin{minipage}[t]{0.5\hsize}    
        \begin{center}
          \includegraphics[width=6.5cm]{fig/iguchi/PMTphoto.jpg}
	\hspace{1cm}[1]2インチPMT H6410
     \end{center}
    \end{minipage}
    %2
      \begin{minipage}[t]{0.5\hsize}    
        \begin{center}
          \includegraphics[width=6.5cm]{fig/iguchi/PMTinner.pdf}
         \hspace{2cm}[2]ラインフォーカス型の内部構造
         \end{center}
        \end{minipage}
     
     \end{tabular}
     \label{PMT12}
    \end{center}
 \end{figure}

PMTを磁場中で使用すると,光電子がPMT内部を進む間にローレンツ力を受けて軌道が曲がり,ダイノードに届かなくなることでゲインが著しく低下する.


\subsection{2インチPMTの測定方法と装置図}
2インチPMTの光電面にNaI(Tl)シンチレータ(直径:57 mm,長さ:58 mm)図\ref{NaIscinti}を取り付け,$\ce{^{60}Co}$と光電面の距離を70 mmに固定し,磁場なしの場合と磁場ありの場合でゲインを測定した.また図\ref{souchi}のように磁極の中心からPMTの光電面までの距離を$r$方向の距離とする.
 
\begin{figure}[h!]
  \begin{center}
    \begin{tabular}{c}
    %1
      \begin{minipage}[h]{0.5\hsize}    
        \begin{center}
          \includegraphics[width=6.5cm]{fig/iguchi/2inchsokutei.pdf}
	\hspace{2cm}[1]測定装置
     \end{center}
    \end{minipage}
    %2
      \begin{minipage}[h]{0.5\hsize}    
        \begin{center}
          \includegraphics[width=6.5cm]{fig/iguchi/soutizu1.pdf}
         \hspace{2cm}[2]装置図
         \end{center}
        \end{minipage}
     
     \end{tabular}
     \label{souchi}
     \caption{2インチPMT装置図}
    \end{center}
 \end{figure}

\begin{figure}[h]
	\centering
	\includegraphics[width=5cm]{fig/iguchi/NaIscinti.jpg}
	\caption{NaI(Tl)シンチレータ}
	\label{NaIscinti}
\end{figure}

 
\subsection{2インチPMTの測定結果}

\begin{figure}[h]
	\centering
	\includegraphics[angle=-90,width=10cm]{fig/iguchi/121550on.pdf}
	\caption{$r=50$ cmのヒストグラム}
	\label{hist50}
\end{figure}

\begin{figure}[h]
	\centering
	\includegraphics[angle=-90,width=10cm]{fig/iguchi/121635on.pdf}
	\caption{$r=35$ cmのヒストグラム}
	\label{hist35}
\end{figure}

ADCを用いて$\ce{^{60}Co}$のスペクトルを取得した.
磁場なしと$r=45$ cmまでのヒストグラムは図\ref{hist50}のように$\ce{^{60}Co}$の2つのピーク(1.17 MeV,1.33 MeV)が見える.
$r=35$ cm(1.22 mT)のときは図\ref{hist35}のようにゲインが大幅に低下し,低い方のピークがバックグラウンドに埋もれてしまっている.$r=25$ cm(4.82 mT)では信号が完全に消失した.
図\ref{plot2inchoff}は縦軸に2インチPMTの磁場中のゲインを,横軸にr方向の距離をとってグラフにした.ただしゲインとは,スペクトルの2つのピークをそれぞれガウシアンでフィットしたものの平均値である.$\ce{^{60}Co}$の2つのピークの形は同じだと考えて,バックグラウンドは一次関数で表せると考えた.さらに縦軸は磁場なしの場合のゲインで規格化した.$r=35$ cm以下では,ピークがフィット出来なかったためプロットしていない.

\begin{figure}[t]
	\centering
		\includegraphics[angle=-90,width=15cm]{fig/iguchi/plot2inchPMT.pdf}
	\caption{2インチPMTの磁場中のゲイン変化}
	\label{plot2inchoff}
\end{figure}
この結果から2インチPMTは$r=35$ cm,$z$方向磁場の強さが1.22 mTで使用出来なくなると結論付ける.

\section{2インチPMTの測定方法と装置図(鉄管あり)}
磁場を遮蔽するため,PMTを鉄管\ref{ironpipe}に挿入して先述と同様の測定を行った.
\begin{figure}[h]
	\centering
		\includegraphics[width=10cm]{fig/iguchi/ironpipe.JPG}
	\caption{鉄管}
	\label{ironpipe}
\end{figure}

\begin{table}[h]
	\centering
	 \begin{tabular}{cccc} \hline
		素材 & 外径 & 内径 & 長さ \\ \hline \hline
		軟鉄 & 76.3 mm & 62.3 mm & 270 mm \\ \hline
	\end{tabular}
	  \caption{鉄管データ表}
	  \label{ironpipedata}
\end{table}

磁力線が鉄管端から内部に入り込む可能性を考慮して,鉄管内部図\ref{ironinnerfigure}のように鉄管端とPMTの光電面との距離が70 mmになるように取り付けた.
鉄は強磁性体であり,磁力線は磁化しやすい物質に吸収されるため,鉄管内部は磁力線が通らなくなり磁場が遮蔽される.

\begin{figure}[tbp]
  \begin{center}
    \begin{tabular}{c}
    %1
      \begin{minipage}[H]{0.6\hsize}    
        \begin{center}
          \includegraphics[width=6.5cm]{fig/iguchi/PMTinFe.pdf}
	\hspace{3cm}[1]鉄管内部
     \end{center}
    \end{minipage}
    %2
      \begin{minipage}[h]{0.4\hsize}    
        \begin{center}
          \includegraphics[width=6.5cm]{fig/iguchi/jibakyusyu.jpg}
         \hspace{3cm}[2]磁性体によって吸収される磁場
         \end{center}
        \end{minipage}
     
     \end{tabular}
     \label{ironinnerfigure}
      \caption{鉄管内部図}
      \end{center}
   \end{figure}
   
\begin{figure}[h]
	\centering
		\includegraphics[width=7cm]{fig/iguchi/2inchinFe.jpg}
	\caption{測定装置(鉄管あり)}
	\label{2inchinFe}
\end{figure}


\subsection{2インチPMTの測定結果(鉄管あり)}

\begin{figure}[h]
	\centering
		\includegraphics[clip,angle=-90,width=10cm]{fig/iguchi/122350fe.pdf}
	\caption{$r=50$ cm(鉄管あり)のヒストグラム}
	\label{hist50fe}
\end{figure}
\begin{figure}[h]
	\centering
		\includegraphics[clip,angle=-90,width=10cm]{fig/iguchi/12237fe.pdf}
	\caption{$r=7$ cm(鉄管あり)のヒストグラム}
	\label{hist7fe}
\end{figure}

\begin{figure}[h]
	\centering
		\includegraphics[angle=-90,width=15cm]{fig/iguchi/bigPMTfit.pdf}
	\caption{2インチPMTでの磁場中のゲイン変化(鉄管あり)}
	\label{bigPMTfit}
\end{figure}

鉄管なしの場合と同様にADCを用いて$\ce{^{60}Co}$のスペクトルを取得した.
鉄管に入れた場合,$r=7$ cmまで図\ref{hist7fe}のように$\ce{^{60}Co}$の2つのピーク(1.17 MeV,1.33 MeV)が見える.
図\ref{bigPMTfit}は,鉄管なしの場合と同様の手法でピークをフィットして得たゲインを規格化してプロットしたものである.$r=20$ cmまでゲインがほとんど低下せず,$r=20$ cm以下ではゲインが低下するが,磁場なしの場合と比べて一割程度低下するだけである.
本実験において2インチPMTは$r=12$ cmに設置する予定なので,鉄管に挿入し磁場を遮蔽することで,本実験でも使用可能である.


\newpage
\section{3/8インチPMT}
四角型PMT(R2248)\cite{pmtR2248}は,本実験でSCtrig用に使用するため,2章の図\ref{fig:device2}のように2インチPMTより磁極に近い位置に,光電面が$r=1.5$ cmとなるように設置される.動作確認実験は早急におこなう必要があり,また磁場の影響の大きさは主にダイノード構造に関係するため,本実験で用いる四角型PMTと同じダイノード構造を持った3/8インチPMT\cite{pmtH3164-10}を用意して磁場中での動作確認をおこなった.

\begin{figure}[htbp]
	\centering
		\includegraphics[width=10cm]{fig/iguchi/miniPMT.jpg}
	\caption{3/8インチPMT H3164-10}
	\label{3/8inch}
\end{figure}

\begin{table}[h]
	\centering
	
	  \begin{tabular}{cccccccc} \hline
		型名& 管径 & 外径 & 長さ & 最大電圧 & 印加電圧 & ゲイン & ダイノード構造 \\ \hline \hline
		H3164-10 & 3/8 インチ & 10.5 mm & 45 mm & -1500 V & -1400 V & $1.0\times10{^{6}}$ &ラインフォーカス型/8段 \\ \hline
	\end{tabular}
	  \caption{3/8インチPMTデータ表}
	  \label{3/8inchPMT}
\end{table}

\begin{table}[h]
	\centering
	  \begin{tabular}{cccccccc} \hline
		型名& 管径 & 外径 & 長さ & 最大電圧 & 印加電圧 & ゲイン & ダイノード構造 \\ \hline \hline
		R2248 & 8 mm(四角型) & 9.8 mm(四角型)& 45 mm & -1500 V & - & $1.1\times10{^{6}}$ &ラインフォーカス型/8段 \\ \hline
	\end{tabular}
	  \caption{角型PMTデータ表}
\end{table}


\subsection{3/8インチPMTの測定方法と装置図}

2インチPMTの動作確認実験と同様の方法で,$\ce{^{22}Na}$と3/8インチPMTの光電面との距離を100 mmで固定し,鉄管に入れた場合と入れない場合でゲインを測定した.
3/8インチPMTはダークカレントが多く線源のスペクトルが見えにくかったため,一つのNaIシンチレータに二つの3/8インチPMTを取り付け,それらの信号を,本実験と同様に同時計測することでダークカレントを落とした.

\begin{figure}[h]
  \begin{center}
    \begin{tabular}{c}
    %1
      \begin{minipage}[h]{0.4\hsize}    
        \begin{center}
          \includegraphics[width=6.5cm]{fig/iguchi/PPMT.jpg}
	\hspace{3cm}[1]3/8インチPMTのコインシデンス
     \end{center}
    \end{minipage}
    %2
      \begin{minipage}[h]{0.6\hsize}    
        \begin{center}
          \includegraphics[width=10cm]{fig/iguchi/miniset.pdf}
         \hspace{3cm}[2]3/8インチPMTの装置図
         \end{center}
        \end{minipage}
     
     \end{tabular}
    \end{center}
   \end{figure}
   
また鉄管は2インチPMTの実験で用いたものと同一のものを使用し,線源はスペクトルのピークの位置がわかりやすい$\ce{^{22}Na}$を使用した.


\subsection{3/8インチPMTの測定結果}

\begin{figure}[h]
	\centering
		\includegraphics[angle=-90,width=10cm]{fig/iguchi/minicoout22.pdf}
	\caption{$r=22$ cmのヒストグラム(鉄管なし)}
	\label{histminicoout22}
\end{figure}

\begin{figure}[h]
	\centering
		\includegraphics[angle=-90,width=10cm]{fig/iguchi/minicoin22.pdf}
	\caption{$r=50$ cmのヒストグラム(鉄管あり)}
	\label{histminicoin22}
\end{figure}

\begin{figure}[h]
	\centering
		\includegraphics[angle=-90,width=10cm]{fig/iguchi/minicoin21.pdf}
	\caption{$r=6$ cmのヒストグラム(鉄管あり)}
	\label{histminicoin21}
\end{figure}

\begin{figure}[h]
	\centering
		\includegraphics[angle=-90,width=10cm]{fig/iguchi/miniPMTgainG.pdf}
	\caption{3/8インチPMTの磁場中でのゲイン変化}
	\label{miniPMTgainG}
\end{figure}

グラフ\ref{miniPMTgainG}は先述の実験と同様の手法で,横軸にr方向の距離,縦軸にゲインをとった.縦軸は磁場なしで測定したときのゲインで規格化した.また1.275MeVのピークは図\ref{histminicoout22}のようにガウシアン+二次関数でフィットした.鉄管に入れない場合,$r=30$ cm(2.36 mT)まではゲインが低下しないが,$r=25$ cm(4.82 mT)付近からゲインが低下し,図\ref{histminicoout22}のように$r=22$ cm(10 mT以下)でゲインが低下しピークが見えなくなった.
鉄管に入れた場合,$r=6$ cmまでゲインが低下せず,図\ref{histminicoin21}のヒストグラムのようにピークがはっきり見える.


\section{3/8インチPMTの測定結果と磁場シミュレーション}
測定結果から3/8インチPMTはコインシデンスを取り,鉄管に入れることで$r=6$ cmまで使用できることがわかった.
しかし本実験では図\ref{honjikken}のように,四角型PMTを外径21.7 mm,内径16.1 mmの鉄管に挿入し,3/8インチPMTを$r=1.5$ cmに設置する.

\begin{figure}[H]
	\centering
		\includegraphics[width=10cm]{fig/iguchi/honjikken.pdf}
	\caption{本実験でのセットアップ}
	\label{honjikken}
\end{figure}

本実験をおこなう前に,上のセットアップでも四角型PMTが使用出来るかを確認する必要がある.そのため,磁場シミュレーションソフト(Femtet)を使用して電磁石の磁場を再現し,本実験で四角型PMTを挿入する鉄管も同様に再現して,鉄管内部の磁場をシミュレーションした.

\subsection{磁場シミュレーション結果}
\begin{figure}[h]
	\centering
		\includegraphics[width=16cm]{fig/iguchi/magnetgraph2.pdf}
	\caption{Femtetで再現した磁場と実測定した磁場}
	\label{magnetgraph2}
\end{figure}

図\ref{magnetgraph2}から確認できるように,Femtetで再現した$z$方向磁場と実際の測定結果を比較すると,ほとんど一致していることからFemtetの磁場シュミレーションは妥当であることがわかる.

\begin{figure}[H]
	\centering
		\includegraphics[width=15cm]{fig/iguchi/Femtetsaigengraph.pdf}
	\caption{Femtetで再現した磁場(鉄管あり)}
	\label{Femtetsaigengraph}
\end{figure}
図\ref{Femtetsaigengraph}の,鉄管内部の$z$方向磁場のシミュレーション結果から,本実験で四角型PMTの光電面が設置される$r=1.5$ cmでの磁場の強さは1.5 mTである.3/8インチPMTが使用可能であった$r=30$ cmでの磁場の強さは2.36 mT,使用不可能になった磁場の強さは約10 mTである.これらの結果を比較すると,シミュレーションで再現した鉄管内部の磁場の方が弱いため,四角型PMTは管径21.7 mmの鉄管に入れることで磁場中でも使用可能である.光電面が$r=2$ cmになるように設置すると図\ref{Femtetsaigengraph}のグラフから,磁場の大きさは約0.1 mTであり,四角型PMTは確実に使用出来る.鉄管内部に入り込む磁場の影響を完全に排除するために,アクリルライトガイドを1 cm延長して,
光電面が$r=2$ cmになるように設置することも考えられる.

\section{まとめ}
2インチPMTは鉄管に挿入し磁場を遮蔽することで$r=7$ cmまで使用可能であったため,本実験で設置する予定である$r=12$ cmでも鉄管に挿入することで使用可能であると結論付ける.また3/8インチPMTは2つのPMTを同時計測し鉄管に挿入することで$r=6$ cmまで使用可能であった.本実験でSCtrig用の四角型PMTを挿入する鉄管の内部磁場を磁場シュミレーションで再現したものと,3/8インチPMTが使用可能であった磁場を比較した結果から,四角型PMTは本実験で設置する予定である$r=1.5$ cmでも鉄管に挿入することで使用可能であると結論付ける.



