\chapter{戦争と平和}

\section{この付録の目的}
戦争と平和の文章を用いて,用いそうな\TeX コマンドのチュートリアルとします。

\section{章構成}

日露戦争の始まって以来、どの雑誌もほとんど戦争の話で持切りのありさまで、あるいは海戦陸戦の実況を報じ、あるいは戦時における人民の心得を論じていたが、これは時節柄もっともな次第であった。しかしそのうち、戦時における心得を論じたものを見るに、多くは戦争と平和とを相反するもののごとくに見なし、戦時には平常と異なった特別の心得方が必要であるかのごとくに説いてあるが、戦争がすんで平和が回復せられたのちに、平和は戦争の反対であると誤解して、戦時に必要な心得をことごとく捨てて顧みぬようなことでもあっては、せっかくの戦勝の利益もその大部はしばらくの間に消えてしまうおそれがある。かような失策を防ぐためには、平生から戦争とは何か、平和とは何かという問題を研究してこれらを明らかにしておかねばならぬ。

%図のサンプル
\begin{figure}[htbp]
	\centering
		\includegraphics[width=5cm]{img/war_and_peace.jpg}
	\caption{戦争と平和はドラマ化されてます}
	\label{warAndPeace}
\end{figure}

 世の中には平和はつねであって、戦争は例外であると思うている人がとかく多いようであるが、世界の歴史を調べてみれば、実際はその反対であることが明らかに知れる。試みに歴史の中から戦争のあった時間だけを除いたとすれば、残りはほとんど何もない。かしこが平和であるときには、ここで戦争があり、甲の所で戦争が終わるころには乙の所で戦争が始まる。全世界を通じていえば、どこにも戦争のないという日は開闢かいびゃく以来おそらく一日もなかろう($d=0$)。一国一国に分けて論ずれば戦争と戦争との間には若干ずつの平和の時代がはさまっているごとくに見えるが、これもていねいに考えてみると決して真の平和ではない。

%簡単な式
\[\sum_{\mathrm{All country}} d_{\mathrm{day}} = 0 \]

その間には必ず砲台を築き、軍艦を造り、できうる限り兵力を整えて、意識的かあるいは無意識的かに次の戦争の準備に全力をつくしているゆえ、機が熟すればささいな口実を種にしてたちまち戦い始める。およそ戦争の芽を含まぬ平和は今日にいたるまでいまだ決して一回もなかったと言うてよろしかろう。
%難しい式

\begin{equation}
	\iint  d_{\mathrm{peace}} = 0 \label{eq:baka}
\end{equation}

さればいわゆる平和なるものはあたかも芝居の幕間のごときもので、つまり式\ref{eq:baka}より、単に次の戦争に対する準備の時期を言い現わす言葉に過ぎぬ。
