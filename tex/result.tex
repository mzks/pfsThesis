\chapter{ポジトロニウムの超微細構造の測定における再現実験を含めた事象選別と結果}\label{result}

\section{再現実験}
超微細構造の測定にはポジトロニウムの寿命を求める必要がある.
そのため,超微細構造の測定に先立ち昨年度の再現実験を行った.

\subsection{実験装置}
昨年度の測定を参考にし,同様のセットアップで行った.
以下に実験装置の概要図を示す.\\
(装置の概要図を入れる)\\

イベントは,まず線源に用いている\ce{^{22}Na}からの1275keVのprompt-$\gamma$線を上方に配置されたNaI(Tl)シンチレータSで,下方に陽電子が出てポジトロニウムを形成し,崩壊する際に放出する$\gamma$線を下に配置してある3つのNaI(Tl)シンチレータを用いて検出する.
