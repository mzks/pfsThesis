\chapter{ポジトロニウムの超微細構造の測定における再現実験を含めた事象選別と結果}\label{result}

\section{再現実験}
超微細構造の測定にはポジトロニウムの寿命を求める必要がある.
そのため,超微細構造の測定に先立ち昨年度の再現実験を行った.

\subsection{実験装置}
昨年度の測定を参考にし,同様のセットアップで行った.
以下に実験装置の概要図を示す.\\
(装置の概要図を入れる)\\

イベントは,まず線源に用いている\ce{^{22}Na}からの1275keVのprompt-$\gamma$線を上方に配置されたNaI(Tl)シンチレータSで,下方に陽電子が出てシリカエアロジェル中の電子と反応してポジトロニウムを形成し,崩壊する際に放出する$\gamma$線を下に配置してある3つのNaI(Tl)シンチレータを用いて検出する.

\subsection{回路図}
全体の回路図は以下のようになっている.\\
(回路の図を貼る)\\

本実験では各シンチレータに入射する$\gamma$線のエネルギーと時刻を測定するためADC,TDCモジュールを用いた.以下に詳細を示す.
\begin{table}[htbp]
	\begin{center}
		\caption{使用したモジュール}
		\begin{tabular}{|l|c|r|r|} \hline
			モジュール & 型番 & シリアルナンバー & 製造会社 \\ \hline \hline
			ADC & V005 & & 豊伸電子 \\ \hline
			TDC & TMC & & REPIC \\ \hline
		\end{tabular}
		\label{module}
	\end{center}
\end{table}

\subsection{discriminatorの閾値}
回路図に示すようにシンチレータSとシンチレータA,B,Cではそれぞれの回路中にあるdiscriminatorで設定されている閾値が異なる.
\begin{description}
	\item[シンチレータSについて]\mbox{}\\
		シンチレータSの閾値highは1275keVの$\gamma$線のみを検出するために用いられる.\\
		highの閾値78.0mVは約900mVのエネルギーに相当する.一方で閾値lowは$\gamma$線の入射時刻を決める役割を持ち,この閾値を下げることで精度を向上させることができる.
		さらにhigh,lowの信号の時間差から事象選別にも利用される.
	\item[シンチレータA,B,Cについて]\mbox{}\\
		シンチレータA,B,CはSよりも低い閾値が設定されている.\\
		これは閾値highの役割が無関係なノイズを落とすためであり,dividerを通る回数によりhighはlowの2倍の閾値となる.
		lowの役割はSと同様$\gamma$線の入射時刻を決めるものである.
\end{description}

\subsection{イベントを得る条件}
イベントは以下のようにして得られる.
\begin{enumerate}
	\item シンチレータSに$\gamma$線が入射し2つある閾値を超える
	\item gate generatorにより幅800nsのcoincidence gateが出力される
	\item シンチレータA,B,Cのいずれかに$\gamma$線が入り,OR回路からdelayで遅延される
	\item Sからの信号がAND回路に入って800ns以内にA(B,C)からの信号が来る
\end{enumerate}

\section{再現実験の結果}

\subsection{シンチレータの較正}
まず,4つのシンチレータの較正直線を求める.
この再現実験では\ce{^{22}Na}のスペクトルをADCを用いて取得し,2つのピークそれぞれ既知の511keVと1275keVに対応するものとしてガウス関数でフィットする.

以下に較正に用いたデータを示す.\\
(較正に用いたデータを貼る)\\
これらのデータにより次の較正直線が得られた.
\begin{eqnarray*}
	E_A{\rm [keV]}=1.79\times {\rm ADC}-174\\
	E_B{\rm [keV]}=1.69\times {\rm ADC}-148\\
	E_C{\rm [keV]}=1.86\times {\rm ADC}-150\\
	E_S{\rm [keV]}=1.86\times {\rm ADC}-145
\end{eqnarray*}
以後,再現実験ではこの較正直線を元にADCとエネルギーの変換を行う.

\subsection{得られたデータ}
本実験で得られたデータを図\ref{fig:dec_t_b4}に示す.
\[
		\left(
			\begin{tabular}{l}
				測定日:2017/12/05  測定時間:27928 sec\\
				run9075  500000 events
			\end{tabular}
		\right)
\]

\begin{figure}[htbp]
	\begin{center}
		\includegraphics[width=10cm]{fig/isb/decay_t.pdf}
		\caption{崩壊時間の分布}
		\label{fig:dec_t_b4}
	\end{center}
\end{figure}

図\ref{fig:dec_t_b4}はシンチレータA,B,Cでの各崩壊時間の分布を足しあげたものである.ここでの崩壊時間はシンチレータSの閾値lowを超えた時刻からシンチレータA,B,Cでの閾値lowを超えた時間の差である.\\
t=0付近に見られる鋭いピークは1つのポジトロニウムの崩壊事象由来のものではないアクシデンタルなイベントが多く含まれるために現れるものである.
また,t=800ns付近でイベントが大幅に減少しているのは,シンチレータSの閾値highを超えたときに出力されるcoincidence gateの幅を800nsとしたためである.

この結果に対し,2種類のカットをかけてバックグラウンド事象を除去しオルソポジトロニウムの崩壊由来の事象を選別する.
\begin{itemize}
	\item low-highカット…光電効果由来1275keVのprompt-$\gamma$線による事象を選別
	\item エネルギーカット…オルソポジトロニウムの崩壊由来の$\gamma$線を選別
\end{itemize}

\subsection{シンチレータSに対する事象選別}
シンチレータSで観測されるパルスは主に図\ref{fig:pulse_diff}に示すような光電効果によるものとコンプトン散乱によるものの2種類あり,それぞれの特徴としては
\begin{itemize}
	\item 光電効果…波高が大きい
	\item コンプトン散乱…波高が小さい
\end{itemize}
であるがどちらも赤点線で示す立ち上がり時間は一定となる.
\begin{figure}[H]
	\begin{center}
		\includegraphics[width=15cm]{fig/isb/pulse_difference.pdf}
		\caption{パルスの違い}
		\label{fig:pulse_diff}
	\end{center}
\end{figure}
このことを利用して光電効果由来の事象のみを選別する.

図\ref{fig:tHL_all}にシンチレータSに設定された2つの閾値を超えた時間の差\\
\begin{center}
	$t_{\rm HL}=t_{\rm high}-t_{\rm low}$
\end{center}
の分布を示す.
\begin{figure}[htbp]
	\begin{tabular}{cc}
		\begin{minipage}{0.5\hsize}
			\begin{center}
			%1枚目
				\includegraphics[width=80mm]{fig/isb/tHL.pdf}
				\caption{$t_{\rm HL}$の分布}
				\label{fig:tHL_all}
			\end{center}
		\end{minipage}
		%2枚目
		\begin{minipage}{0.5\hsize}
			\begin{center}
				\includegraphics[width=80mm]{fig/isb/tHL_cut.pdf}
				\caption{t=30nsまでの$t_{\rm HL}$分布}
				\label{fig:tHL_zoom}
			\end{center}
		\end{minipage}
	\end{tabular}
\end{figure}

$t_{\rm HL}$=10ns付近に光電効果由来のパルスによるピークが立ち,その後にはコンプトン散乱由来のパルスにより,なだらかなイベント数の減少が確認できる.
図\ref{fig:tHL_zoom}は図\ref{fig:tHL_all}において,$t_{\rm HL}$=0から30ns付近を拡大したものである.\\
これより事象選別に利用する$t_{\rm HL}$の条件として\\
\begin{center}
	$8\leq$$t_{\rm HL}$$\leq12$
\end{center}
を採用した.
\subsubsection{シンチレータSに対する事象選別結果}
図\ref{fig:dec_t_cut}にlow-highカット後の崩壊時間分布を示す.\\
\begin{figure}[htbp]
	\begin{center}
		\includegraphics[width=10cm]{fig/isb/decay_t_cut}
		\caption{low-highカット後の崩壊時間分布}
		\label{fig:dec_t_cut}
	\end{center}
\end{figure}
low-highカットによりコンプトン散乱由来の事象を含む約76\%のイベントがカットされた.
