\chapter{ポジトロニウムの超微細構造の測定における再現実験を含めた事象選別と結果}\label{result}

\section{再現実験}
超微細構造の測定にはポジトロニウムの寿命を求める必要がある.
そのため,超微細構造の測定に先立ち昨年度の再現実験を行った.

\subsection{実験装置}
昨年度の測定を参考にし,同様のセットアップで行った.
以下に実験装置の概要図を示す.\\
(装置の概要図を入れる)\\

イベントは,まず線源に用いている\ce{^{22}Na}からの1275keVのprompt-$\gamma$線を上方に配置されたNaI(Tl)シンチレータSで,下方に陽電子が出てシリカエアロジェル中の電子と反応してポジトロニウムを形成し,崩壊する際に放出する$\gamma$線を下に配置してある3つのNaI(Tl)シンチレータを用いて検出する.

\subsection{回路図}
全体の回路図は以下のようになっている.\\
(回路の図を貼る)\\

本実験では各シンチレータに入射する$\gamma$線のエネルギーと時刻を測定するためADC,TDCモジュールを用いた.以下に詳細を示す.
\begin{table}[htb]
	\begin{center}
		\caption{使用したモジュール}
		\begin{tabular}{|l|c|r|r|} \hline
			モジュール & 型番 & シリアルナンバー & 製造会社 \\ \hline \hline
			ADC & V005 & & 豊伸電子 \\ \hline
			TDC & TMC & & REPIC \\ \hline
		\end{tabular}
		\label{module}
	\end{center}
\end{table}

\subsection{discriminatorの閾値}
回路図に示すようにシンチレータSとシンチレータA,B,Cではそれぞれの回路中にあるdiscriminatorで設定されている閾値が異なる.
\begin{description}
	\item[シンチレータSについて]\mbox{}\\
		シンチレータSの閾値highは1275keVの$\gamma$線のみを検出するために用いられる.\\
		highの閾値76.0mVは約900mVのエネルギーに相当する.一方で閾値lowは$\gamma$線の入射時刻を決める役割を持ち,この閾値を下げることで精度を向上させることができる.
		さらにhigh,lowの信号の時間差から事象選別にも利用される.
	\item[シンチレータA,B,Cについて]\mbox{}\\
		シンチレータA,B,CはSよりも低い閾値が設定されている.\\
		これは閾値highの役割が無関係なノイズを落とすためであり,dividerを通る回数によりhighはlowの2倍の閾値となる.
		lowの役割はSと同様$\gamma$線の入射時刻を決めるものである.
\end{description}

\subsection{イベントを得る条件}
イベントは以下のようにして得られる.
\begin{enumerate}
	\item シンチレータSに$\gamma$線が入射し2つある閾値を超える
	\item gate generatorにより幅800nsのcoincidence gateが出力される
	\item シンチレータA,B,Cのいずれかに$\gamma$線が入り,OR回路からdelayで遅延される
	\item Sからの信号がAND回路に入って800ns以内にA(B,C)からの信号が来る
\end{enumerate}

\section{再現実験の結果}

