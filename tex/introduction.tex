\chapter{序論}\label{introduction}

ポジトロニウムとは,電子と陽電子が電磁相互作用によって束縛されている系である.
またポジトロニウムには系の全スピンによりふたつの状態,
パラポジトロニウムとオルソポジトロニウムがある.
このふたつの状態のエネルギー差を超微細構造といい,
量子電磁力学により約203 GHz(=0.84 meV)と計算されている.
しかし超微細構造は理論値と測定値に約3.5 $\sigma$のずれがあり,
標準模型を超えた新しい物理が存在する可能性がある.

本研究室での昨年度の卒業研究では,
大気中でのオルソポジトロニウムの寿命の測定を行った.
本研究では昨年度の実験から発展し,
ポジトロニウムに磁場をかけ超微細構造を測定するために,
実験装置の作製,装置の評価,実験のシミュレーションを行った.

